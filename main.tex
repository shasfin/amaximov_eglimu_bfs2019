\documentclass{article}
%encoding
%--------------------------------------
\usepackage[utf8]{inputenc}
\usepackage[T1]{fontenc}
%--------------------------------------
 
%German-specific commands
%--------------------------------------
\usepackage[ngerman]{babel}
\usepackage{csquotes}
%--------------------------------------

%Pictures
%--------------------------------------
\usepackage{graphicx}
\graphicspath{ {./Pictures/} }
\usepackage{tikz}
\usepackage{subcaption}
\usepackage{float}
\usepackage{wrapfig}
%--------------------------------------

%math
%--------------------------------------
\usepackage{amsmath}
\usepackage{amssymb}
\usepackage{amsfonts}
%--------------------------------------

\usepackage{multicol}
\usepackage[shortlabels]{enumitem}

%Aufgaben
%--------------------------------------
\newtheorem{aufgabe}{Aufgabe}[section]
\newtheorem{definition}{Definition}[section]
\newtheorem{beispiel}{Beispiel}[section]
%--------------------------------------

%Listings
%--------------------------------------
\usepackage{listings}
\usepackage{xcolor}
 
\definecolor{codegreen}{rgb}{0,0.6,0}
\definecolor{codegray}{rgb}{0.5,0.5,0.5}
\definecolor{codepurple}{rgb}{0.58,0,0.82}
\definecolor{backcolour}{rgb}{0.95,0.95,0.92}
 
\lstdefinestyle{mystyle}{
    backgroundcolor=\color{backcolour},   
    commentstyle=\color{codegreen},
    keywordstyle=\color{magenta},
    numberstyle=\tiny\color{codegray},
    stringstyle=\color{codepurple},
    basicstyle=\ttfamily\footnotesize,
    breakatwhitespace=false,         
    breaklines=true,                 
    captionpos=b,                    
    keepspaces=true,                 
    numbers=left,                    
    numbersep=5pt,                  
    showspaces=false,                
    showstringspaces=false,
    showtabs=false,                  
    tabsize=2,
}
 
\lstset{style=mystyle}
%--------------------------------------


\title{Die Breitensuche -- eine Einführung für Schüler der Gymnasialstufe}
\author{eglimu, amaximov}
\date{November 2019}

\begin{document}

\maketitle

\tableofcontents

\section{Einführung}
\begin{figure}[h]
    \centering
    \includegraphics[width=\linewidth]{Pictures/Tram1.PNG}
    \caption{Ein Ausschnitt der Tram- und Buslinien in Zürich}
\end{figure}

Alice steht nach der Schule an der Haltestelle ''Kantonsschule'' in Zürich und möchte einen ganz bestimmten Kugelschreiber einkaufen gehen. Ihr Ziel ist es möglichst wenige Zwischenhaltestellen zu besuchen, bevor sie eine erreicht, in deren Nähe ein Laden mit dem genau diesem Kugelschreiber im Sortiment steht. Sie beginnt also zu überlegen: ''Die benachbarten Haltestellen sind\\ ''ETH/Universitätsspital'', ''Platte'' und ''Kunsthaus'', aber bei keiner dieser Stationen finde ich meinen Kugelschreiber. Nun kann ich von diesen drei Haltestellen wiederum für alle Benachbarten überlegen ob es dort den Kugelschreiber gibt, denn das sind genau jene Stationen die ich mit nur einem Zwischenstopp erreichen kann.'' Auf diese Art überlegt Alice weiter bis sie eine Haltestelle findet in deren Nähe der Kugelschreiber verkauft wird. \\

Das Problem der Suche nach dem Kugelschreiber können wir als Graph formalisieren, indem wir für jede Haltestelle einen Knoten zeichnen und jeweils die Benachbarten mit einer Kante verbinden. Alice beginnt also bei ihrem Startknoten ''Kantonsschule'' und durchläuft dann die restliche Knoten aufsteigend geordnet nach dem Abstand zu diesem Startknoten. Erinnere dich, dass wir den Abstand von zwei Knoten in einem Graphen definiert haben als die Anzahl Kanten des kürzesten Weges vom ersten Knoten zum zweiten. Das von Alice durchgeführte Vorgehen entspricht der Breitensuche, einem Algorithmus mit welchem wir uns in den folgenden 2 Lektionen befassen werden.\\

Nachdem wir im ersten Quartal die Grundlagen der Graphentheorie sowie das Implementieren von einfachen Algorithmen in Python gelernt haben, wollen wir jetzt diese beiden Themen vereinen indem wir mit der Breitensuche einen Algorithmus für Graphen betrachten. Am Ende dieser zwei Lektionen werdet ihr die Breitensuche für einen beliebigen Graphen selbst durchführen können und den Algorithmus auch in Python implementieren können. Ihr werdet erkennen wie vielfältig die Breitensuche angewendet werden kann und den Algorithmus und sein Programm an einige verschiedene Problemstellungen anpassen können.

Wir beginnen im nächsten Abschnitt damit, das genaue Vorgehen des Algorithmus zuerst in Worten und dann anhand einer Folge von Bildern durchzugehen. Zudem implementieren wir die Breteinsuche in Python. Anhand der zugehörigen Aufgaben lernst du bereits verschiedene mögliche Anwendungen kennen.  In Abschnitt 3 geht es darum die Breitensuche zu verwenden um kürzeste Pfade zwischen zwei bliebigen Knoten in einem Graphen zu finden.

\section{Breitensuche}
\subsection{Der Algorithmus graphisch}
Wir betrachten wieder das Tramliniennetz von Zürich und zeichnen den Ausschnitt mit den 10 Knoten
\begin{multicols}{2}
\begin{itemize}
    \item {\bf{B}}: Bellevue
    \item {\bf{C}}: Central
    \item {\bf{E/U}}: ETH/Universitätsspital
    \item {\bf{HB}}: Haldenbach
    \item {\bf{HE}}: Haldenegg
    \item {\bf{HP}}: Hottingerplatz
    \item {\bf{KH}}: Kunsthaus
    \item {\bf{KS}}: Kantonsschule
    \item {\bf{N}}: Neumarkt
    \item {\bf{P}}: Platte

\end{itemize}
\end{multicols}
\noindent vereinfacht als Graphen auf. Wir haben die Knoten bewusst alphabetisch aufgelistet, denn immer, wenn wir alle Nachbarn einer Haltestelle betrachten müssen, werden wir dies in alphabetischer Reihenfolge tun.
\begin{figure}[H]
    \centering
    \includegraphics[width=\linewidth]{Pictures/Tram2.PNG}
    \caption{Ein Ausschnitt der Tram- und Buslinien in Zürich, als Graph formalisiert}
\end{figure}
Damit wir mit der Breitensuche beginnen können, müssen wir einen Startknoten auswählen. Wir wählen hier wie im Beispiel in der Einführung die Kantonsschule, also den Knoten {\bf{KS}} als Startknoten und schreiben ihn gleich auf eine Warteliste. Wir arbeiten also wiedereinmal mit dem ''First In, First Out''-Prinzip, denn wir gehen in jedem Schritt so vor, dass wir den ersten Knoten von der Warteliste entfernen und dafür seine noch unbesuchten Nachbarn der Warteliste anhängen. Im folgenden verwenden wir 3 Farben für die Knoten um die Breitensuche zu veranschaulichen. Blaue Knoten sind jene, die die Breitensuche noch nicht erreicht hat. Orange Knoten sind die besuchten Knoten, welche noch auf der Warteliste stehen. Graue Knoten sind jene, die wir bereits wieder von der Warteliste entfernt haben. Überlege anhand des Einführungsbeispiels, warum es so wichtig ist die besuchten Knoten zu markieren!

\begin{figure}[H]
    \centering
    \begin{subfigure}[h]{0.45\textwidth}
    \includegraphics[width=\textwidth]{Pictures/BS/BFSB1.PNG}
    \caption{Die Breitensuche beginnt beim Startknoten {\bf{KS}}. Im ersten Schritt schreiben wir diesen auf unsere Warteliste und markieren ihn als besucht.}
    \label{fig:BS1}
    \end{subfigure}
    \vspace{5mm}
    \qquad
    \begin{subfigure}[h]{0.45\textwidth}
    \raggedleft
    \includegraphics[width=\textwidth]{Pictures/BS/BFSB.PNG}
    \caption{Wir entfernen {\bf{KS}} aus der Warteliste und färben ihn darum grau. Als nächstes wollen wir die direkt benachbarten Haltestellen betrachten, wir schreiben sie also alphabetisch geordnet auf die Warteliste und markieren sie als besucht.}
    \label{fig:BS2}
    \end{subfigure}
    \vspace{5mm}
    \begin{subfigure}[h]{0.45\textwidth}
    \raggedright
    \includegraphics[width=\textwidth]{Pictures/BS/BFSB2.PNG}
    \caption{ Der erste Knoten auf der Warteliste war {\bf{E/U}}, wir entfernen ihn und hängen seine zwei unbesuchten Nachbarn {\bf{HB}} und {\bf{HE}} der Warteliste an.}
    \end{subfigure}
    \vspace{5mm}
    \qquad
    \begin{subfigure}[h]{0.45\textwidth}
    \raggedleft
    \includegraphics[width=\textwidth]{Pictures/BS/BFSB3.PNG}
    \caption{Der erste Knoten auf der Warteliste war {\bf{KH}}, wir entfernen ihn und hängen seine drei unbesuchten Nachbarn {\bf{B}}, {\bf{HP}} und {\bf{N}} der Warteliste an.} 
    \end{subfigure}
    \end{figure}
\begin{figure}[H]\ContinuedFloat
    \begin{subfigure}[h!]{0.45\textwidth}
    \centering
    \includegraphics[width=\textwidth]{Pictures/BS/BFSB4.PNG}
    \caption{Als letzten Nachbarn von {\bf{KS}} können wir {\bf{P}} von der Warteliste entfernen. Er hat keine weiteren unbesuchten Nachbarn.} 
    \end{subfigure}
    \vspace{5mm}
    \qquad
    \begin{subfigure}[h]{0.45\textwidth}
    \centering
    \includegraphics[width=\textwidth]{Pictures/BS/BFSB5.PNG}
    \caption{Als nächstes stehen in der Warteschlange die Nachbarn von {\bf{E/U}}. Wir entfernen zuerst {\bf{HB}} und müssen keine Knoten der Warteliste anhängen.}
    \end{subfigure}
    \vspace{5mm}
    \begin{subfigure}[h]{0.45\textwidth}
    \centering
    \includegraphics[width=\textwidth]{Pictures/BS/BFSB6.PNG}
    \caption{Wir entfernen {\bf{HE}} von der Warteliste, und hängen seinen unbesuchten Nachbarn {\bf{C}} der Warteliste an.}
    \end{subfigure}
    \qquad
    \begin{subfigure}[h]{0.45\textwidth}
    \centering
    \includegraphics[width=\textwidth]{Pictures/BS/BFSB7.PNG}
    \caption{Wir entfernen {\bf{B}} aus der Warteliste. Der Knoten hat keine unbesuchten Nachbarn.}
    \end{subfigure}
\end{figure}
\begin{figure}[H]\ContinuedFloat
    \begin{subfigure}[h]{0.45\textwidth}
    \centering
    \includegraphics[width=\textwidth]{Pictures/BS/BFSB8.PNG}
    \caption{Wir entfernen {\bf{HP}} aus der Warteliste. Der Knoten hat keine unbesuchten Nachbarn.}
    \end{subfigure}
    \qquad
    \begin{subfigure}[h]{0.45\textwidth}
    \centering
    \includegraphics[width=\textwidth]{Pictures/BS/BFSB9.PNG}
    \caption{Als letzten Nachbarn von {\bf{KH}} können wir {\bf{N}} von der Warteliste entfernen. Die Nachbarn von {\bf{N}} sind alle bereits als besucht markiert. }
    \end{subfigure}
\end{figure}
\begin{figure}[H]\ContinuedFloat
     \centering
     \begin{subfigure}[h]{0.45\textwidth}
    \centering
    \includegraphics[width=\textwidth]{Pictures/BS/BFSB10.PNG}
    \caption{Zuletzt entfernen wir {\bf{C}} aus der Warteliste. Seine Nachbarn wurden alle bereits einmal besucht und die Warteliste ist jetzt leer. Der Algorithmus endet also an dieser Stelle.}
    \end{subfigure}
\end{figure}
Die Breitensuche besucht die Knoten also in der Reihenfolge 
$$ \text{ \bf{KS,E/U,KH,P,HB,HE,B,HP,N,C}} $$

\subsection{Der Algorithmus in Worten}
Wir wollen nun den Suchalgorithmus, welchen wir oben graphisch an einem Beispiel aufgezeichnet haben, allgemein formalisieren. In einer ersten Version geben wir dem Algorithmus keine Abbruchsbedingung, das heisst die Breitensuche läuft weiter, bis alle möglichen Knoten einmal besucht wurden. Um uns jeweils zu merken, welche Knoten wir als nächstes betrachten wollen, verwenden wir eine Warteliste. Um uns zu merken, welche Knoten wir bereits betrachtet haben, markieren wir diese als besucht. So können wir vermeiden, dass ein Knoten mehrmals bearbeitet wird.\\

 \noindent {\bf{Input:}} Ein Graph $G=(V,E)$ und ein Startknoten $s \in V$.

\noindent {\bf{Output:}} Die Knoten von $G$ aufsteigend geordnet nach Abstand zum Startknoten $s$.

\begin{enumerate}
    \item Schreibe den Startknoten $s$ auf die Warteliste und markiere ihn als besucht.
    \item Entferne den ersten Knoten von der Warteliste und prüfe für jeden Nachbarn, ob er bereits als besucht markiert wurde.
    \item Falls nicht, hänge ihn der Warteliste an und markiere ihn als besucht.
    \item Wiederhole die Schritte 2 und 3 bis keine Knoten mehr auf der Warteliste sind.
\end{enumerate}

\begin{aufgabe} \label{newinitial}
In welcher Reihenfolge besucht die Breitensuche die Haltestellen, wenn wir statt bei der Kantonsschule beim Bellevue starten?
\end{aufgabe}

\begin{aufgabe} \label{directed}
Wir betrachten immer noch den Graphen in Abbildung \ref{fig:BS1}. Nun
treten bei verschiedenen Trams technische Störungen auf, so dass die Strecken Bellevue - Platte und Central - Hottingerplatz jeweils nur noch in dieser einen Richtung befahren werden können. Zeichne den gerichteten Graphen (mit den gleichen 10 Knoten), der zu dieser Situation passt. In welcher Reihenfolge besucht die Breitensuche jetzt die Knoten, wenn wir wieder beim Bellevue starten? 
\end{aufgabe}


Alice und Bob spielen das Wikipedia-Spiel. Es funktioniert wie folgt: Alice gibt Bob zwei Wikipedia-Artikel vor und dieser muss durch das Klicken von möglichst wenigen Links von einem Artikel zum anderen kommen. Wir können Wikipedia als Graphen darstellen, indem wir für jeden Artikel einen Knoten zeichnen und für einen Link in Artikel $A$ zu Artikel $B$ eine gerichtete Kante von $A$ nach $B$. Der tatsächliche Graph von allen deutschen Wikipedia-Artikeln hat so fast 2.5 Millionen Knoten und noch viel mehr Kanten. Wir betrachten nur einen sehr kleinen Ausschnitt davon mit den folgenden Artikeln:
\begin{multicols}{2}
\begin{itemize}
    \item {\bf{B}}: Bellevue
    \item {\bf{BS}}: Breitensuche
    \item {\bf{D}}: Deutschland
    \item {\bf{E}}: Eisenbahn
    \item {\bf{G}}: Graph
    \item {\bf{K}}: Knoten
    \item {\bf{L}}: Laufzeit
    \item {\bf{S}}: Schweiz
    \item {\bf{SBZ}}: Strassenbahn Zürich
    \item {\bf{U-B}}: U-Bahn
    \item {\bf{Z}}: Zürich
\end{itemize}

\end{multicols}
mit dem zugehörigen gerichteten Graphen:
\begin{figure}[h!]
    \centering
    \includegraphics[scale=0.5]{Pictures/Wikipedia.PNG} 
    \caption{Die Verlinkung von Wikipedia-Artikeln als Graph.}
    \label{fig:my_label}
\end{figure}

\begin{aufgabe} \label{Wikipedia}
Alice sagt zu Bob: ''Öffne den Wikipedia-Artikel über die Breitensuche und finde den Artikel über die Strassenbahnen von Zürich.'' Wenn Bob nur die 11 aufgelisteten Artikel zur Verfügung hat, kann er dann den gewünschten Artikel erreichen? Verwende die Breitensuche um die Frage zu beantworten.
\end{aufgabe}

In Aufgabe \ref{Wikipedia} haben wir eine Abbruchsbedingung in Form eines gesuchten Knotens. Wir können also der Breitensuche als Input zusätzlich zum Graphen und Startknoten noch einen Zielknoten geben und den Algorithmus abbrechen, sobald der Zielknoten als besucht markiert wurde.

Jetzt wäre es natürlich schön, wenn wir den Algorithmus so abändern könnten, dass er uns zusätzlich sagt, wieviele Klicks Bob mindestens braucht und was eine mögliche optimale Lösung wäre. Doch bevor wir dieses Problem lösen, wollen wir zuerst die Breitensuche in Python implementieren.

\subsection{Der Algorithmus in Python}
Wir wollen ein Programm für die Breitensuche schrieben und dieses gleich an unserem Tramlinienbeispiel testen.
Da wir uns bei der Breitensuche jeweils für jeden Knoten alle seine Nachbarn merken (bzw. auf die Warteliste schreiben) müssen, bietet es sich an die ''Liste der Nachbarn''- Darstellung für die Graphen zu verwenden.
In Python implementieren wir einen Graphen also am besten als eine Menge von Listen. Der Graph in Abbildung \ref{fig:BS1} wäre zum Beispiel:

\begin{figure}[H]
    \centering
    \includegraphics[scale=0.8]{Pictures/ListeDerNachbarn.PNG}
    \caption{Tramliniengraph in der ''Liste der Nachbarn''-Darstellung}
    \label{fig:Tram2}
\end{figure}

Tippe das Programm in Abbildung \ref{fig:Breitensuche} ab. Vergleiche die Ausgabe mit der Reihenfolge der Knoten die wir im Abschnitt 2.1 gefunden haben.

\begin{figure}[H]
    \centering
    \includegraphics[scale=0.8]{Pictures/Breitensucheprog.PNG}
    \caption{Breitensuche in Python}
    \label{fig:Breitensuche}
\end{figure}

Wir nennen das Programm ''bfs'' für ''breadth first search'' (englischer Name der Breitensuche). 

\begin{aufgabe} \label{ProgFragen} Beantworte folgende Fragen zum Programm in Abbildung \ref{fig:Breitensuche}
\begin{itemize} 
    \item Wofür brauchen wir die Listen ''queue'' und ''visited''?
    \item Welche Zeilen im Programm entsprechen welchem Schritt des Algortihmus in Abschnitt 2.1?
\end{itemize}
\end{aufgabe}

\paragraph{Die Laufzeit} hängt bei einem Algorithmus für Graphen davon ab, wieviele Knoten und Kanten dieser hat. Falls die Breitensuche durch den ganzen Graphen $G=(V,E)$ läuft, so durchlaufen wir die while-Schlaufe für jeden Knoten einmal, also insgesammt $|V|$-mal. In der for-Schlaufe betrachten wir jeweils jede ausgehende Kante, insegsamt durchlaufen wir sie also maximal (bei einem ungerichteten Graphen) $2|E|$-mal. Die Laufzeit der einzelnen Schritte ist immer $\mathcal{O}(1)$. Wir erhalten eine Gesamtlaufzeit von $\mathcal{O}(|V|+|E|)$. Die Zeit, welche der Computer benötigt um die Breitensuche durchzuführen, wächst also linear in der Anzahl der Knoten und Kanten des Graphens an.

\paragraph{Der Speicherplatzverbrauch} der Breitensuche ist $\mathcal{O}(|V|)$ da wir alle bisher besuchten Knoten speichern. Sobald du mehrere Algorithmen für Graphen kennst kannst du sie anhand ihrer Laufzeiten und ihrem Speicherplatzverbrauch miteinander vergleichen.

\begin{aufgabe}  \label{WikiProgAufg} {\bf{*}}
Ändere das Programm ''bfs'', so dass es abbricht sobald ein gesuchter Knoten gefunden wurde und überprüfe damit deine Lösung von Aufgabe \ref{Wikipedia}. Nenne das neue Programm ''bfs\_goal''.
\end{aufgabe}

\paragraph{Lernaufgabe}
Alice und Bob haben eine Datenbank ('Familien') zur Verfügung in welcher für jeden Menschen jeweils die Eltern und die Kinder aufgeführt sind. Sie geht 6 Generationen zurück. Nun fragen sie sich ob sie eine gemeinsame Ur-ur-ur-ur-Grossmutter haben. 
\begin{enumerate}
    \item Wie können sie die Daten als ungerichteten Graphen darstellen?
    \item Wie können sie die Antwort zu ihrer Frage mit Hilfe des Programms ''bfs\_goal'' finden? Beschreibe deine Idee in Worten. (noch nicht programmieren)
\end{enumerate}
Nun wollen sie ein Programm welches ihnen statt der Liste aller besuchten Knoten als Ausgabe einfach mitteilt ob sie eine gemeinsame Ur-ur-ur-ur-Grossmutter haben und zwar in Form einer Ausage wie: '' Der Startknoten und der Endknoten sind in \dots '' oder ''Der Startknoten und der Endknoten sind nicht in \dots''
\begin{enumerate}[resume]
    \item Was kommt in die Lücke des gewünschten Outputs?
    \item Schreibe das Programm und teste es an einem kleinen Graphen.
\end{enumerate}

\subsection{Kontrollfragen}
\begin{enumerate}
    \item Wie würdest du einem Mitschüler, welcher diesen ersten Teil der Stunde verpasst hat, in wenigen Sätzen erklären, was die Breitensuche ist?
    \item Du kennst bereits viele verschiedene Zusammenhänge, welche sich als Graphen darstellen lassen. Überlege dir eine ''Alltagssituation'', bei der du die Breitensuche verwenden könntest.
\end{enumerate}
Welche der folgenden Schüler haben recht? Welche nicht? Begründe!
\begin{enumerate}[resume]
    \item David sagt:''Es gibt bestimmt Graphen bei denen ein Knoten im Laufe der Breitensuche mehr als einmal in der Warteliste auftaucht.''
    \item Fred behauptet: ''Ob ich einen bestimmten Knoten in einem zusammenhängeden Graphen finden kann hängt nicht davon ab welchen Startknoten ich gewählt habe.''
    \item * Gregory meint: '' Ich kann einen zusammenhängenden Graphen mit beliebig vielen Knoten zeichnen und den Startknoten so wählen, dass nie mehr als ein Knoten auf der Warteliste steht.
\end{enumerate}

\newpage
\newpage

\section{Kürzeste Wege}
\input{Kuerzeste_Wege.tex}
\newpage

\section{Zusammenfassung}
Du hast mit der Breitensuche deinen ersten Algorithmus für Graphen kennengelernt. Der Algorithmus nimmt als Eingabe einen Graphen und einen Startknoten in diesem Graphen und durchläuft dann alle Knoten in der gleichen Zusammenhangskompnente wie der Startknoten geordnet nach aufsteigendem Abstand vom Startknoten. Man kann den Algorithmus auch so abändern, dass man als Eingabe zusätzlich einen gesuchten Zielknoten gibt und die Suche abbricht sobald dieser gefunden wurde.

Du hast gelernt, dass du den Algorithmus leicht abändern kannst, um den Abstand zwischen zwei Knoten zu ermitteln. Dafür musst du für jeden besuchten Knoten merken, wir weit er vom Startknoten entfernt war.

Wenn du dir für jeden merkst, warum er besucht wurde (als Nachbar von welchem Knoten), kannst du mithilfe von Backtracking den kürzesten Pfad zwischen dem Startknoten und einen beliebigen Knoten herausfinden. Wenn der Startknoten gleich bleibt, musst du nur das Backtracking neu machen.

Ausserdem hast du gesehen, wie sich viele unteschiedliche Probleme als Graphprobleme modellieren und lösen lassen. Zum Beispiel, hast du gesehen, dass Menschen, Haltestellen, Wikipediaartikel, Fliesen, Wörter und Spielkästchen als Knoten dienen können, und dass ein Staubsauger-Roboter, der zu seiner Aufladestation muss, sich im Grunde genommen nicht von einem Mädchen unterscheidet, welches unter ihren Bekannten eine von einem Panda gebissene Person sucht, oder von einem Jungen, der mit zwei gezinkten Münzen ein Schlangenspiel gewinnen will.
\newpage

\newpage
\section{Beispiellösungen}
\paragraph{ Aufgabe \ref{newinitial}:} Wenn wir bei der Station ''Bellvue'' starten ist die Reihenfolge: $$ \text{\bf{B,KH,HP,KS,N,E/U,P,C,HB,HE}}$$

\paragraph{Aufgabe \ref{directed}:} Wir betrachten jetzt den folgenden Graphen:\\
\begin{figure}[H]
    \centering
    \includegraphics[scale=0.3]{Pictures/directedTram.PNG} 
    \caption{Der gerichtete Graph.}
    \label{fig:my_label}
\end{figure}
Die Reihenfolge in welcher die Breitensuche die Knoten besucht ist:
$$ \text{\bf{B,KH,HP,KS,E/U,P,HB,HE,C,N}}$$

\paragraph{Aufgabe \ref{Wikipedia}}: Wir beginnen die Breitensuche beim Wikipedia-Artikel über die Breitensuche; also bei dem Knoten {\bf{BS}}. Die Reihenfolge ist dann
$$ \text{\bf{BS,D,G,K,L,E,S,U-B,Z,SBZ}}.$$
Bob kann den gewünschten Artikel also finden und wir könen die Breitensuche abbrechen bevor wir den Knoten {\bf{B}} besucht haben.

\paragraph{Aufgabe \ref{ProgFragen}:} Die Liste ''queue'' enthält genau unsere Warteliste, in ''visited'' fügen wir Schritt für Schritt die besuchten Knoten hinzu, so dass wir am Ende mit ''return visited'' die Knoten in der gewünschten Reihenfolge zurückgeben können.

\begin{figure}[H]
    \centering
    \includegraphics[scale=0.3]{Pictures/Loesung24.jpg}
    \caption{Die 4 Schritte des Algorithmus im Programm}
    \label{fig:ProgLoes}
\end{figure}

\paragraph{Aufgabe \ref{WikiProgAufg}:} Wir implementieren den Graphen der Wikipedia-Artikel in der ''Liste der Nachbarn''-Darstellung und passen das Programm an (Abbildung \ref{fig:my_Prog2Loes}). Falls der Start- und der Zielknoten gleich sind können wir die Suche sofort abbrechen. Ansonsten überprüfen wir jeweils nachdem wir einen Nachabrn als besucht markiert haben, ob dieser der gesuchte Knoten ist. Fall ja können wir die for-Schleife mit break beenden. Damit auch die while-Schleife endet leeren wir noch die Warteliste.

\begin{figure}[H]
    \centering
    \includegraphics[scale=0.6]{Pictures/WikiProgLoes.PNG}
    \caption{Das Programm mit Zielknoten als Input und einer Abbruchsbedingung}
    \label{fig:my_Prog2Loes}
\end{figure}

\paragraph{Lernaufgabe} 
\begin{enumerate}
    \item Als Knoten nehmen wir alle Menschen. Wir verbinden je zwei Knoten wenn einer ein direkter Nachkomme vom anderen ist. Der Graph stellt also einfach einen Stammbaum dar.
    \item Sie können ''print bfs\_goal (Familien,'Alice','Bob')'' aufrufen. Falls bei der Ausgabe an letzter Stelle 'Bob' steht so haben sie eine gemeinsame Ur-ur-ur-ur-Grossmutter. Wenn die Breitensuche 'Bob' nicht findet, so sind Alice und Bob nicht so nahe verwandt.
    \item der gleichen Zusammenhangskomponente.
    \item Das angepasste Programm:
    \begin{figure}[H]
        \centering
        \includegraphics[scale=0.4]{Pictures/components.PNG}
        \caption{Wir rufen bfs\_goal auf und prüfen ob es den Zielknoten finden konnte.}
        \label{fig:my_components}
    \end{figure}
\end{enumerate}

\paragraph{Kontrollfragen}
\begin{enumerate}
    \item -
    \item - 
    \item Die Aussage von David ist falsch. Wir prüfen immer ob wir einen Knoten bereits besucht haben, bevor wir ihn auf die Warteliste setzen.
    \item Fred's Aussage stimmt nur für ungerichtete Graphen. In einem gerichteten Graphen kann es sein, dass man einen Knoten nicht findet, obwohl er sich in der gleichen Zusammenhangskomponente wie der Startknoten befindet.
    \item Gregory's Aussage ist richtig, er kann den Knoten ganz links als Startknoten nehmen in Abbildung \ref{fig:Gregory}.
    \begin{figure}[H]
        \centering
        \includegraphics[width=\textwidth]{Pictures/Kontrollfragen.PNG}
        \caption{Gregory's Graph}
        \label{fig:Gregory}
    \end{figure}
\end{enumerate}

\paragraph{Augabe \ref{aufgabe_panda_gebissen}}
\begin{enumerate}[(a)]
\item Die Knoten sind die genannten Leuten (Alice, Bob, Charlie, David, Elisabeth, Fred, Gregory, Hannah, Jakob, Lucy). Zwischen zwei Knoten gibt es eine Kante, wenn die entsprechenden Leuten sich irgendwoher schon kennen. Der Graph sieht dann so aus (wenn wir die Knoten mit dem ersten Buchstaben des Vornamens beschriften). Der Graph ist ungerichtet, weil wenn Person A Person B kennt, dann kennt Person B Person A auch.
    \begin{figure}[H]
        \centering
        \includegraphics[width=\textwidth]{Pictures/SP/panda_gebissen_graph.png}
        \caption{Freundschaftsgraph}
        \label{fig:Gregory}
    \end{figure}

\item Wir markieren in grau die Knoten, die schon besucht worden sind und in Orange diejenigen, die sich in der Warteschlange befinden. In der Warteschlange schreiben wir unter dem Knoten auch den Abstand vom Startknoten auf.
\begin{figure}[H]
    \centering
    \begin{subfigure}[h]{0.45\textwidth}
    \includegraphics[width=\textwidth]{Pictures/SP/panda_gebissen_0.png}
    \end{subfigure}
    \vspace{5mm}
    \qquad
    \begin{subfigure}[h]{0.45\textwidth}
    \raggedleft
    \includegraphics[width=\textwidth]{Pictures/SP/panda_gebissen_1.png}
    \end{subfigure}
    \vspace{5mm}
    \centering
    \begin{subfigure}[h]{0.45\textwidth}
    \raggedright
    \includegraphics[width=\textwidth]{Pictures/SP/panda_gebissen_2.png}
    \end{subfigure}
    \qquad
    \begin{subfigure}[h]{0.45\textwidth}
    \raggedleft
    \includegraphics[width=\textwidth]{Pictures/SP/panda_gebissen_3.png}
    \end{subfigure}
    \begin{subfigure}[h]{0.45\textwidth}
    \includegraphics[width=\textwidth]{Pictures/SP/panda_gebissen_4.png}
    \end{subfigure}
    \vspace{5mm}
    \qquad
    \begin{subfigure}[h]{0.45\textwidth}
    \raggedleft
    \includegraphics[width=\textwidth]{Pictures/SP/panda_gebissen_5.png}
    \end{subfigure}
    \vspace{5mm}
    \centering
    \begin{subfigure}[h]{0.45\textwidth}
    \raggedright
    \includegraphics[width=\textwidth]{Pictures/SP/panda_gebissen_6.png}
    \end{subfigure}
    \qquad
    \begin{subfigure}[h]{0.45\textwidth}
    \raggedleft
    \includegraphics[width=\textwidth]{Pictures/SP/panda_gebissen_7.png}
    \end{subfigure}
    \begin{subfigure}[h]{0.45\textwidth}
    \includegraphics[width=\textwidth]{Pictures/SP/panda_gebissen_8.png}
    \end{subfigure}
\end{figure}
\item 
\begin{lstlisting}[language=Python, caption={Freundschaftsgraph in der Liste der Nachbarn Darstellung.}]
friends = {"A":["B", "G"],
        "B":["A", "C", "D"],
        "C":["B", "E", "F"],
        "D":["B"],
        "E":["C", "F"],
        "F":["C", "E", "G", "J"],
        "G":["A", "F", "H", "J"],
        "H":["G", "J"],
        "J":["F", "G", "H", "L"],
        "L":["J"]}
\end{lstlisting}
\item Dieser Graph ist klein und wir können durch Ausprobieren bestimmen, dass die zwei Knoten, die am weitesten auseinander sind, sind \"D\" und \"L\" und der Abstand zwischen diesen zwei Knoten ist 5.
\begin{lstlisting}[language=Python]
print(length_shortest_path(friends, "D", "L"))
5
\end{lstlisting}
Wir können aber auch ein Programm schreiben, das uns die Länge vom längsten kürzesten Pfad (den Diameter) ausrechnet. Dafür werden wir zunächst Programm \ref{lst:length_shortest_path} so umschreiben, dass es die Längen von allen kürzesten Pfaden vom Startknoten aus ausgibt. Die zwei grössten Änderungen sind, dass wir die Abstände nun auch in \texttt{visited} reinschreiben, so dass wir sie ausgeben können, und dass wir die Suche nicht mehr abbrechen, wenn wir den Endknoten gefunden haben (deswegen geben wir dem Programm auch keinen Endknoten mit).
\begin{lstlisting}[language=Python, caption={Programm, welches die Abstände vom Startknoten zu allen anderen erreichbaren Knoten ausrechnet}]
def length_all_shortest_paths_from(graph, initial):
    queue = [(initial, 0)]
    visited = {initial: 0}
    while queue:
        (node, length) = queue.pop(0)
        newlength = length + 1
        neighbours = graph[node]
        for neighbour in neighbours:
            if neighbour not in visited.keys():
                queue.append((neighbour, newlength))
                visited[neighbour] = newlength
    return visited
\end{lstlisting}
Jetzt können wir dieses Programm verwenden, um den Diameter auszurechnen. Wir gehen durch alle Knoten im Graphen durch und rufen \texttt{length\_all\_shortest\_paths\_from} auf. Das gibt uns für jeden Knoten die Abstände zu allen anderen erreichbaren Knoten im Graphen. Dann nehmen wir das Maximum über all diesen kürzesten Pfade und somit bestimmen wir den Diameter.
\begin{lstlisting}[language=Python, caption={Programm, welches den Diameter von einem Graphen bestimmt}]
def diameter(graph):
    maximum = (0, "from", "to")
    for node in graph.keys():
        all_shortest_from = length_all_shortest_paths_from(graph, node)
        farthest_from_node = max(all_shortest_from, key=all_shortest_from.get)
        if (all_shortest_from[farthest_from_node] > maximum[0]):
            maximum = (all_shortest_from[farthest_from_node], node, farthest_from_node)
    return maximum
\end{lstlisting}
Wenn wir dieses Programm ausführen, erfahren wir, dass der Diameter vom Fraundschaftsgraphen 5 ist, und dass der Abstand zwischen \texttt{D} und \texttt{L} genau diesen Wert hat.
\begin{lstlisting}
print(diameter(friends))
(5, 'D', 'L')
\end{lstlisting}
\end{enumerate}

\paragraph{Aufgabe \ref{aufgabe_shortest_path}}
Zuerst verwenden wir die Breitensuche, um alle Elternknoten zu ermitteln. Dann führen wir \texttt{backtracking} aus.
\begin{lstlisting}[language=Python]
def shortest_path(graph, initial, goal):
    parents = bfs_with_parent(graph, initial)
    return backtracking(parents, goal)
\end{lstlisting}

\paragraph{Aufgabe \ref{aufgabe_erde_mars_graph}}
\begin{enumerate}[(a)]
    \item Die Knoten sind die Wörter aus der Liste. Es gibt genau dann eine Kante zwischen zwei Wörter, wenn die Wörter sich um genau einen Buchstaben unterscheiden, wie zum Beispiel \texttt{MARS} und \texttt{MARK}. Der Graph ist ungerichtet, weil wenn sich ein Wort A um einen Buchstaben von einem Wort B unterscheidet, dann unterscheited sich auch Wort B vom Wort A um einen Buchstaben.
    \item
    \begin{figure}[H]
    \centering
    \includegraphics[width=0.6\textwidth]{Pictures/SP/bank_zins.png}
\end{figure}
\end{enumerate}

\paragraph{Aufgabe \ref{aufgabe_erde_mars_neighbours}}
\begin{enumerate}[(a)]
    \item
    \begin{lstlisting}[language=Python]
    words = {"ERDE":["EIDE", "ENDE", "ERLE"],
        "EIDE":["EILE", "EINE", "ENDE", "ERDE"],
        "EINE":["EIDE", "EILE", "EINS"],
        "EINS":["EINE", "PINS", "ZINS"],
        "ZINS":["EINS", "PINS", "ZINK", "ZINN"],
        "ZINK":["ZINS", "ZINN", "ZANK", "ZICK"],
        "ZANK":["BANK", "SANK", "ZACK", "ZINK"],
        "ZACK":["ZANK", "PACK", "ZICK", "SACK"],
        "PACK":["ZACK", "SACK", "PARK"],
        "PARK":["PACK", "PARD", "MARK"],
        "MARK":["PARK", "MARS"],
        "MARS":["MARK", "MAUS"],
        "MAUS":["MARS", "HAUS"],
        "HAUS":["MAUS", "HASS"],
        "HASS":["HAUS", "BASS"],
        "BASS":["HASS"],
        "PARD":["PARK"],
        "BAND":["SAND", "BANK"],
        "SAND":["BAND", "SANK"],
        "SACK":["SANK", "ZACK", "PACK"],
        "BANK":["BAND", "SANK", "ZANK"],
        "SANK":["BANK", "ZANK", "SACK", "SAND"],
        "SINN":["ZINN"],
        "ZINN":["SINN", "ZINK", "ZINS"],
        "ZICK":["ZINK", "ZACK"],
        "PINS":["ZINS","EINS"],
        "EILT":["EILE"],
        "EILE":["EINE", "EILT", "EULE"],
        "EULE":["EILE", "ERLE"],
        "ERLE":["EULE", "ERDE"],
        "ENDE":["ERDE", "EIDE", "ENTE"],
        "ENTE":["ENDE"]}
    \end{lstlisting}
    \item
    \begin{lstlisting}[language=Python]
    print(shortest_path(words, "ERDE", "MARS"))
    ['ERDE', 'EIDE', 'EINE', 'EINS', 'ZINS', 'ZINK', 'ZANK', 'ZACK', 'PACK', 'PARK', 'MARK', 'MARS']
    \end{lstlisting}
    \item Die Idee ist, dass wir zuerst für den ersten Buchstaben alle 26 Möglichkeiten ausprobieren und Wörter aus der Referenzliste zu den Nachbarn hinzufügen. Dann machen wir dasselbe für den zweiten, dritten und vierten Buchstaben.
    \begin{lstlisting}[language=Python]
    def compute_neighbours(nodes, node):
    neighbours = []
    for c1 in string.ascii_uppercase:
        neighbour = c1+node[1:4]
        if neighbour != node:
            if neighbour in nodes:
                neighbours.append(neighbour)
                
    for c2 in string.ascii_uppercase:
        neighbour = node[0]+c2+node[2:4]
        if neighbour != node:
            if neighbour in nodes:
                neighbours.append(neighbour)
    
    for c3 in string.ascii_uppercase:
        neighbour = node[0:2]+c3+node[3:4]
        if neighbour != node:
            if neighbour in nodes:
                neighbours.append(neighbour)
                
    for c4 in string.ascii_uppercase:
        neighbour = node[0:3]+c4
        if neighbour != node:
            if neighbour in nodes:
                neighbours.append(neighbour)
    return neighbours
    \end{lstlisting}
    Dieses Programm kann eleganter gemacht werden, indem man eine Schleife anstatt von den vier ähnlichen Blöcke verwendet.
    
    \item In \texttt{bfs\_with\_parent} rufen wir \texttt{compute\_neighbours} auf. Anstatt vom Graphen geben wir die Liste der Knoten herum. Das \texttt{backtracking} Programm muss gar nicht verändert werden.
    \begin{lstlisting}[language=Python]
    def bfs_with_parent_neighbours(nodes, initial):
    queue = [initial]
    visited = {initial:None}
    while queue:
        node = queue.pop(0)
        neighbours = compute_neighbours(nodes, node)
        for neighbour in neighbours:
            if neighbour not in visited:
                queue.append(neighbour)
                visited[neighbour] = node
    return visited
    
def shortest_path_neighbours(nodes, initial, goal):
    parents = bfs_with_parent_neighbours(nodes, initial)
    return backtracking(parents, goal)
    \end{lstlisting}
    
    \item Ja, es ist möglich. Analog zu \texttt{compute\_neighbours} können wir alle 4 Möglichkeiten (nach vorne, nach hinten, nach links, nach rechts) ausprobieren und überprüfen, ob wir sie zu den Nachbarn hinzufügen können. Aber hier die Bedingung ist eine andere: die Blöcke, die in der Möbelliste (die die Wände auch aufzählt) auftauchen, werden nicht zu den Nachbarn hinzugefügt.
    
    \end{enumerate}

\paragraph{Aufgabe \ref{aufgabe_schlangenspiel}}
\begin{enumerate}[(a)]
    \item Die Knoten sind die Kästchen, wo die Zahlen stehen. Zwischen Knoten A und Knoten B gibt es eine Kante, wenn Kästchen B aus dem Kästchen A mit einem Würfelwurf erreicht werden kann (inklusive Schlagen- und Leiterzauber). Der Graph ist gerichtet, weil man nicht zurück oder in der Gegenrichtung von Zauberkästchen gehen kann.
    \item Eine mögliche Darstellung ist:
    \begin{lstlisting}[language=Python]
    snakes_and_ladders = {
        0: [1, 3, 5, 11],
        1: [3, 5, 11],
        2: [3, 5, 6, 11],
        3: [1, 5, 6, 11],
        4: [1, 5, 6, 8],
        5: [1, 6, 8, 9],
        6: [1, 3, 8, 9],
        7: [3, 8, 9, 11],
        8: [3, 9, 11],
        9: [3, 11],
        10: [11],
        11: []
    }
    \end{lstlisting}
    
    \item
    \begin{lstlisting}[language=Python]
    print(shortest_path(snakes_and_ladders, 0, 11))
    [0, 11]
    \end{lstlisting}
    Bob muss eine 4 Würfeln, dann hat er beim ersten Wurf gewonnen.
    
    \item Wir können die Liste der Nachbarn so umschreiben, dass wir die gewürfelte Zahl mitgeben. Dann können wir diese Zahl als Teil vom Nachbarn herumreichen.
    \item Wir brauchen zusätzlich ein Wörterbuch, welches die Zauberfelder (Schlangen und Leiter) auf die effektiven Felder übersetzt. Auf dem kleinen Spielfeld würde das Wörterbuch so aussehen.
    \begin{lstlisting}[language=Python]
        magic_squares = {
            2: 5,
            4: 11,
            7: 1,
            10: 3
        }
    \end{lstlisting}
    Um die Nachbarn zu ermitteln, addieren wir die gewürfelte Zahl und dann übersetzen wir sie mit dem Wörterbuch, um "Zauberfelber" richtig zu behandeln.
    
    Wir müssen jetzt bestimmen, was passiert, wenn man eine zu hohe Zahl würfelt. Das einfachste ist den Spieler in einem solchen Fall gewinnen zu lassen. Dafür mussen wir noch ein paar "Zauberfelder" einfügen.
    \begin{lstlisting}[language=Python]
        magic_squares = {
            2: 5,
            4: 11,
            7: 1,
            10: 3,
            12: 11,
            13: 11,
            14: 11,
            15: 11
        }
    \end{lstlisting}
    Dieses Programm gibt manche Nachbarn doppelt aus, aber bei der Breitensuche, wo wir den \texttt{visited}-Vektor haben, wird es keinen Effekt auf die Korrektheit haben.
    \begin{lstlisting}[language=Python]
    def compute_neighbours(magic_dict, node):
    neighbours = []
    for i in [1, 2, 3, 4]:
        neighbour = node + i
        if neighbour in magic_dict.keys():
            neighbour = magic_dict[neighbour]
        neighbours.append(neighbour)
    return neighbours
    \end{lstlisting}{}
    
    \item Zuerst modifizieren wir \texttt{compute\_neighbours} so dass wir nicht nur die Nachbarn berechnen sondern auch die gewürfelten Zahlen nicht verlieren.
    \begin{lstlisting}[language=Python]
def compute_neighbours_and_dice(magic_dict, node):
    neighbours = []
    for i in [1, 2, 3, 4]:
        neighbour = node + i
        if neighbour in magic_dict.keys():
            neighbour = magic_dict[neighbour]
        neighbours.append((neighbour, i))
    return neighbours
    \end{lstlisting}
    
    Dann schreiben wir \texttt{bfs\_with\_parent} um, so dass wir die gewürfelten Zahlen nicht verlieren.
    \begin{lstlisting}
def bfs_with_parent_and_dice(magic_dict, initial):
    queue = [initial]
    visited = {initial:(None, None)}
    while queue:
        node = queue.pop(0)
        neighbours_and_die = compute_neighbours_and_dice(magic_dict, node)
        for (neighbour, dice) in neighbours_and_die:
            if neighbour not in visited:
                queue.append(neighbour)
                visited[neighbour] = (node, dice)
    return visited
    \end{lstlisting}
    
    Dieses Mal müssen wir das \texttt{backtracking} Programm umschreiben, weil wir anstatt von den Knoten die gewürfelten Zahlen ausgeben wollen.
    \begin{lstlisting}[language=Python]
def backtracking_dice(parents_and_dice, goal):
    reversed_path = []
    (node, dice) = parents_and_dice[goal]
    while node != None:
        reversed_path.append(dice)
        (node, dice) = parents_and_dice[node]
    return list(reversed(reversed_path))
    \end{lstlisting}
    
    Jetzt können wir wie gewohnt alle Teile zusammensetzen.
    \begin{lstlisting}[language=Python]
def shortest_path_neighbours_and_dice(magic_dict, initial, goal):
    parents_and_dice = bfs_with_parent_and_dice(magic_dict, initial)
    return backtracking_dice(parents_and_dice, goal)
    \end{lstlisting}
    
    Wir sind endlich bereit, das grosse Schlangenspiel zu lösen.
    \begin{lstlisting}[language=Python]
magic_squares_big = {
    39: 2,
    15: 5,
    17: 22,
    14: 35,
    56: 36,
    42: 55,
    26: 45,
    32: 13,
    9: 11,
    51: 30,
    60: 59,
    61: 59,
    62: 59,
    63: 59,
}

print(shortest_path_neighbours_and_dice(magic_squares_big, 0, 59))
[1, 4, 4, 3, 3, 4, 4]
    \end{lstlisting}
    Es ist möglich, das grosse Schlangenspiel in 7 Würfe zu gewinnen, wenn man jedes Mal die richtige Zahl würfelt.
\end{enumerate}












\end{document}
